Start with the automatically reduced photometry. I will use {\tt NAME} as the name of the object, {\tt TARGDATE-TARGDATE} as the range of dates for the science images (images where the target was visible), and {\tt TEMPDATE} as the epoch(s) of the reference images. {\tt ID} refers to the ID number of a specific file. {\tt IN} refers to the instrument prefix (kb for SBIG, fl for Sinistro, fs for Spectral). {\tt TEMPTEL} is the instrument prefix for the template image (kb, fl, fs, or SDSS). If you don't give this option, it will select the same instrument given with -T.

\section{Prepare the Reference Images}
\subsection{LCOGT References}
\begin{enumerate}

\item Look through your reference images and choose the best one for each camera--filter combination. Make a note of their ID numbers.

{\tt ds9 \& } \\
{\tt lscloop.py -n NAME -e TEMPDATE -s checkwcs}

\item Mark your chosen images as references.

{\tt lscloop.py -n NAME -e TEMPDATE -d ID -s template}

\item Run cosmic ray rejection on the reference images.

{\tt lscloop.py -n NAME -e TEMPDATE --filetype 4 -s cosmic}

\item Generate PSFs for the reference images. I recommend doing this interactively for each image (using {\tt --show}), since these PSFs will affect all of your subtractions.

{\tt lscloop.py -n NAME -e TEMPDATE --filetype 4 -s psf --show}

\end{enumerate}

\subsection{SDSS References}
\begin{enumerate}

\item Choose a set of science images that includes one image with each camera--filter combination used. Then run the following command once for each of those images, using the ID numbers to choose individual frames. This will take a while.

{\tt lscloop.py -n NAME -e TARGDATE -d ID -s ingestsloan }

Make a note of the {\tt TEMPDATE} for each SDSS frame you download.

\item Generate PSFs for the SDSS images. I usually get good results using a FWHM of $\sim 5$ pixels for SDSS.

{\tt ds9 \& } \\
{\tt lscloop.py -n NAME -e TEMPDATE --filetype 4 -s psf --show --fwhm 5 --use-sextractor }

\item Copy the variance images to the right place for use as cosmic ray masks.

{\tt lscloop.py -n NAME -e TEMPDATE --filetype 4 -s cosmic }

\end{enumerate}

\section{Do the Image Subtraction}

\begin{enumerate}
\item Run cosmic ray rejection on all the science images. This will take a while.

{\tt lscloop.py -n NAME -e TARGDATE-TARGDATE -s cosmic}

\item In the mean time, check to make sure all the science images have good PSFs from the automatic pipeline.

{\tt lscloop.py -n NAME -e TARGDATE-TARGDATE -s checkpsf}

Generate a new PSF for any images you marked as bad. I usually get good results using a FWHM of $\sim 7$ pixels, but this depends on the seeing.

{\tt lscloop.py -n NAME -e TARGDATE-TARGDATE -b psf -s psf --show --fwhm 7}

\item Once all the cosmic ray rejection is done (for science and reference images), run the subtraction. This will take a while.

{\tt lscloop.py -n NAME -e TARGDATE-TARGDATE --tempdate TEMPDATE --normalize t -T IN --temptel TEMPTEL --fixpix -s diff}

If you want, look over the results. Make sure to choose Frame $>$ Tile on DS9.

{\tt lscloop.py -n NAME -e TARGDATE-TARGDATE --filetype 3 -s checkdiff}
\end{enumerate}

\section{Extract the Photometry}
\begin{enumerate}

\item First copy the convolved PSFs over.

{\tt lscloop.py -n NAME -e TARGDATE-TARGDATE --filetype 3 -s psf}

\item Then calculate the instrumental magnitudes. Use a flat background for subtracted images.

{\tt lscloop.py -n NAME -e TARGDATE-TARGDATE --filetype 3 -x 1 -y 1 -c -s psfmag }

If you want, look over the results and redo any bad fits.

{\tt lscloop.py -n NAME -e TARGDATE-TARGDATE --filetype 3 -s checkmag}

\item Calculate the zero point and color term for each set of filters.

{\tt lscloop.py -n NAME -e TARGDATE-TARGDATE --filetype 3 -f landolt --field apass -s zcat }\\
{\tt lscloop.py -n NAME -e TARGDATE-TARGDATE --filetype 3 -f sloan -s zcat }

\item Calculate the apparent magnitudes for each set of filters.

{\tt lscloop.py -n NAME -e TARGDATE-TARGDATE --filetype 3 -f landolt -s mag }\\
{\tt lscloop.py -n NAME -e TARGDATE-TARGDATE --filetype 3 -f sloan -s mag }

\end{enumerate}